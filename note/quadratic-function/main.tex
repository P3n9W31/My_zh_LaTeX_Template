
\documentclass[final]{beamer}

\usepackage[scale=1.24]{beamerposter} % Use the beamerposter package for laying out the poster


% Chinese Support,If you don't need it, just comment it.
\usepackage{ctex}
\setCJKmainfont{STSong}

\usetheme{confposter} % Use the confposter theme supplied with this template

\setbeamercolor{block title}{fg=ngreen,bg=white} % Colors of the block titles
\setbeamercolor{block body}{fg=black,bg=white} % Colors of the body of blocks
\setbeamercolor{block alerted title}{fg=white,bg=dblue!70} % Colors of the highlighted block titles
\setbeamercolor{block alerted body}{fg=black,bg=dblue!10} % Colors of the body of highlighted blocks
% Many more colors are available for use in beamerthemeconfposter.sty

%-----------------------------------------------------------
% Define the column widths and overall poster size
% To set effective sepwid, onecolwid and twocolwid values, first choose how many columns you want and how much separation you want between columns
% In this template, the separation width chosen is 0.024 of the paper width and a 4-column layout
% onecolwid should therefore be (1-(# of columns+1)*sepwid)/# of columns e.g. (1-(4+1)*0.024)/4 = 0.22
% Set twocolwid to be (2*onecolwid)+sepwid = 0.464
% Set threecolwid to be (3*onecolwid)+2*sepwid = 0.708

\newlength{\sepwid}
\newlength{\onecolwid}
\newlength{\twocolwid}
\newlength{\threecolwid}
\setlength{\paperwidth}{48in} % A0 width: 46.8in
\setlength{\paperheight}{36in} % A0 height: 33.1in
\setlength{\sepwid}{0.024\paperwidth} % Separation width (white space) between columns
\setlength{\onecolwid}{0.22\paperwidth} % Width of one column
\setlength{\twocolwid}{0.464\paperwidth} % Width of two columns
\setlength{\threecolwid}{0.708\paperwidth} % Width of three columns
\setlength{\topmargin}{-0.5in} % Reduce the top margin size
%-----------------------------------------------------------

\usepackage{graphicx}  % Required for including images

\usepackage{booktabs} % Top and bottom rules for tables

%----------------------------------------------------------------------------------------
%	TITLE SECTION 
%----------------------------------------------------------------------------------------

\title{Quadratic Function} % Poster title

\author{matematika.pl} % Author(s)

\institute{2015} % Institution(s)

%----------------------------------------------------------------------------------------

\begin{document}

\addtobeamertemplate{block end}{}{\vspace*{2ex}} % White space under blocks
\addtobeamertemplate{block alerted end}{}{\vspace*{2ex}} % White space under highlighted (alert) blocks

\setlength{\belowcaptionskip}{2ex} % White space under figures
\setlength\belowdisplayshortskip{2ex} % White space under equations

\begin{frame}[t] % The whole poster is enclosed in one beamer frame

\begin{columns}[t] % The whole poster consists of three major columns, the second of which is split into two columns twice - the [t] option aligns each column's content to the top

\begin{column}{\sepwid}\end{column} % Empty spacer column

\begin{column}{\onecolwid} % The first column

%----------------------------------------------------------------------------------------
%	OBJECTIVES
%----------------------------------------------------------------------------------------

\begin{alertblock}{Objectives}
Objectives for today:
\begin{itemize}
\item 介绍一下你自己。
\item Introducing specific vocabulary.
\item Quick revision of quadratic function.
\item Factorising Quadratics.
\item Proving Vieta's formulas.
\item Carrying out gained knowledge by working out some word problems.
\end{itemize}

\end{alertblock}

%----------------------------------------------------------------------------------------
%	QUICK REVISION
%----------------------------------------------------------------------------------------

\begin{block}{Quick Revision}

\textbf{Forms of Quadratic Function}
\begin{itemize}
\item $f(x) = ax^2+bx+c$ is called the \textbf{standard form}.
\item $f(x) = a(x-x_1)(x-x_2)$ is called the \textbf{factored form}, where $x_1$ and $x_2$ are the roots of the quadratic function.
\item $f(x) = a(x-h)^2+k$ is called the \textbf{vertex form}.
\end{itemize}

\textbf{Delta $\Delta$}\\*
$\Delta$ determines tells us how many solutions quadratic equation have:
$$\text{number of solutions}=
\begin{cases}
2 &\text{when } \Delta > 0\\
1 &\text{when } \Delta = 0\\
0 &\text{when } \Delta < 0
\end{cases}
$$


\textbf{The Quadratic Formula}
$$x = \frac{-b\pm \sqrt{\Delta}}{2a}$$

\textbf{Graph of Quadratic Function}

\end{block}

%------------------------------------------------

\begin{figure}
\includegraphics[width=0.8\linewidth]{1.jpg}
\caption{Graph of $f(x)=ax^2|_{\{0.1, 0.3, 1.0, 3.0\}}$}
\end{figure}

%----------------------------------------------------------------------------------------

\end{column} % End of the first column

\begin{column}{\sepwid}\end{column} % Empty spacer column

\begin{column}{\twocolwid} % Begin a column which is two columns wide (column 2)

\begin{columns}[t,totalwidth=\twocolwid] % Split up the two columns wide column

\begin{column}{\onecolwid}\vspace{-.6in} % The first column within column 2 (column 2.1)

%----------------------------------------------------------------------------------------
%	MATERIALS
%----------------------------------------------------------------------------------------

\begin{block}{Factorising a Quadratic}

Factorising a quadratic means putting it into two brackets, and is useful if you're trying to draw a graph of a quadratic solve a quadratic equation. It's pretty easy if $a=1$ (in $ax^2+bx+c$ form), but can be a real pain otherwise.
\newline
\newline
In order to factorise a quadratic you should follow steps outlined below:

\begin{enumerate}
\item Rearrange the equation into the standard $ax^2+bx+c$ form.
\item Write down two brackets: $(x\ \ \ )(x\ \ \ )$
\item Find two numbers that multiply to give 'c' and add or subtract to give 'b' (ignoring signs).
\item Put the numbers in brackets and choose their signs.
\end{enumerate}

\end{block}

%----------------------------------------------------------------------------------------

\end{column} % End of column 2.1

\begin{column}{\onecolwid}\vspace{-.6in} % The second column within column 2 (column 2.2)

%----------------------------------------------------------------------------------------
%	P
%----------------------------------------------------------------------------------------

\begin{block}{Factorising- Tasks}
1. Factorise $x^2-x-12$.
\[\]
\[\]
\[\]
\[\]
2. Solve $x^2-8=2x$ by factorising.

\end{block}

%----------------------------------------------------------------------------------------

\end{column} % End of column 2.2

\end{columns} % End of the split of column 2 - any content after this will now take up 2 columns width

%----------------------------------------------------------------------------------------
%	IMPORTANT To REMEMBER
%----------------------------------------------------------------------------------------

\begin{alertblock}{Myth of Delta $\Delta$}

It's commonly believed that in order to work out roots of a quadratic function you must count $\Delta$ and use other previously established formulas. However this is untrue since factorising in many cases is as good or even better than simply counting $\Delta$.

\end{alertblock} 

%----------------------------------------------------------------------------------------

\begin{columns}[t,totalwidth=\twocolwid] % Split up the two columns wide column again

\begin{column}{\onecolwid} % The first column within column 2 (column 2.1)

%----------------------------------------------------------------------------------------
%	EXAMPLE OF FACTORISATION
%----------------------------------------------------------------------------------------

\begin{block}{Example of Factorisation}
Solve $x^2+4x-21=0$ by factorising.

$$x^2+4x-21=(x\ \ \ \ \ )(x\ \ \ \ \ )$$
$1$ and $21$ multiply to give $21$ - and add or subtract to give $22$ and $20$.\\*
$3$ and $7$ multiply to give $21$ - and add or subtract to give $10$ and \textbf{$4$}.

$$x^2+4x+21 = (x+7)(x-3)$$

And solving the equation:
$$(x+7)(x-3)=0$$
we get
$$x=-7,\ \ \ x=3$$

\end{block}

%----------------------------------------------------------------------------------------

\end{column} % End of column 2.1

\begin{column}{\onecolwid} % The second column within column 2 (column 2.2)

%----------------------------------------------------------------------------------------
%	PROOF OF VIETA'S FORMULAS
%----------------------------------------------------------------------------------------

\begin{block}{ Proof of Vieta's Formulas}
Let's prove that:
$$x_1 + x_2 = \frac{-b}{a}$$
When $\Delta$ is positive we have two roots:
$$x_1 = \frac{-b-\sqrt{\Delta}}{2a},\ \ \ x_2 = \frac{-b+\sqrt{\Delta}}{2a}$$

Substituting for $x_1$ and $x_2$ respectively, we receive:
					
$$x_1 + x_2 = \frac{-b-\sqrt{\Delta}}{2a} + \frac{-b+\sqrt{\Delta}}{2a} =$$
$$ = \frac{(-b-\sqrt{\Delta}) + (-b+\sqrt{\Delta})}{2a} = \frac{-2b}{2a} = \frac{-b}{a}$$
                    
The same we could do with another pattern, which state that $x_1 x_2 = \frac{c}{a}$, but proving this is going to be your task in next section.

\end{block}

%----------------------------------------------------------------------------------------

\end{column} % End of column 2.2

\end{columns} % End of the split of column 2

\end{column} % End of the second column

\begin{column}{\sepwid}\end{column} % Empty spacer column

\begin{column}{\onecolwid} % The third column

%----------------------------------------------------------------------------------------
%	CONCLUSION
%----------------------------------------------------------------------------------------

\begin{block}{Vieta's Formulas- Task}
1. Prove that $$x_1x_2 = \frac{c}{a}$$
\[\]
\[\]
\[\]
\[\]
\[\]

\end{block}


%----------------------------------------------------------------------------------------
%	ACKNOWLEDGEMENTS
%----------------------------------------------------------------------------------------

\setbeamercolor{block title}{fg=red,bg=white} % Change the block title color

\begin{block}{Glossary}

\begin{table}
\vspace{2ex}
\begin{tabular}{l l l l}
\toprule
\textbf{verb} & \textbf{noun} & \textbf{meaning}\\
\midrule
add & addition & $+$ \\
subtract & subtraction & $-$ \\
multiply & multiplication & $\cdot$ \\
divide & division & $\div$ \\
solve & solution & getting answer \\
substitute & substitution & $t=x^2$ \\



\bottomrule
\end{tabular}
\caption{Word Formation}
\end{table}


\end{block}

\setbeamercolor{block alerted title}{fg=black,bg=norange} % Change the alert block title colors
\setbeamercolor{block alerted body}{fg=black,bg=white} % Change the alert block body colors

\begin{alertblock}{Some Necessary and Useful Vocabulary}

\begin{itemize}
\item (n.) sign $\rightarrow$ $+$ or $-$
\item (n.) equation $\rightarrow something = 0$ 
\item (n.) factor $\rightarrow$ two multiplied factors give result
\item (v.) factorise $\rightarrow$ putting into brackets
\item (n.) coefficient $\rightarrow$ a constant number i.e. $a$, $b$, $c$ in a pattern $ax^2+bx+c$
\item (n.) quadratic function $\rightarrow$ $f(x) = ax^2+bx+c$
\item (n.) root $\rightarrow$ $\sqrt{sth}$ or solution of quadratic equation
\item (n.) formula $=$ pattern
\end{itemize}

\end{alertblock}


%----------------------------------------------------------------------------------------

\end{column} % End of the third column

\end{columns} % End of all the columns in the poster

\end{frame} % End of the enclosing frame

\end{document}
