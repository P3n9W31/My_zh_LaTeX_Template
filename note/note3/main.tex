\documentclass[%
aps,
prl,
12pt, 
twocolumn,
preprint,
reprint,
amsmath,
amssymb
]{revtex4-1}

\usepackage{graphicx}
\usepackage{dcolumn}
\usepackage{bm}
\usepackage{natbib}
\usepackage{zhlipsum}
\usepackage{lipsum}
\usepackage{float}
\usepackage{booktabs}
\usepackage{multirow}
\usepackage{ctex}
\graphicspath{{./figures/}}
%
\begin{document}

\title{\LaTeX~Note} 
\author{作者一} 
\affiliation{xx大学 xx学院} 
\author{作者二} 
\affiliation{yy大学 yy学院} 

\begin{abstract}
\zhlipsum[1]

\end{abstract}

%
\maketitle %生成标题
%
\section{第一节}

\zhlipsum[1]

% 图片
\begin{figure} [h]
\includegraphics[width=6cm]{Fig1.png}
\caption{This is a bear}
\label{fig1}
\end{figure}

%%%%%
\section{第二节}
%%%%%%
\zhlipsum[1]
%%%%%%%
\section{第三节}
%%%%%%%%

\begin{enumerate}
\item 孤山寺北贾亭西,水面初平云脚低。
\item 几处早莺争暖树,谁家新燕啄春泥。
\item 乱花渐欲迷人眼,浅草才能没马蹄。
\item 最爱湖东行不足,绿杨阴里白沙堤。 

\end{enumerate}


%%%%
\section{第四节}
%%%%

\lipsum[1]

% 公式
\begin{align*}
I(X;Y) = & h(X)-h(X|Y)\\
= & \frac{1}{2}\log(2\pi e \sigma_x^2) - \frac{1}{2}\log(2\pi e \sigma_x^2(1-\rho^2))\\
= & \frac{1}{2}\log(\frac{2\pi e}{1-\rho^2})
\end{align*}

表\ref{tab1}是随便写的一个表格。

% 表格
\begin{table}[h!]
\caption{Sample table title}
\label{tab1} 
      \begin{tabular}{cccc}
        \hline
           & B1  &B2   & B3\\ \hline
        A1 & 0.1 & 0.2 & 0.3\\
        A2 & ... & ..  & .\\
      \end{tabular}
\end{table}


%%%%%%%%%%%%
\section{第五节}
%%%%%%%%%%%%
\lipsum[1]

\end{document}